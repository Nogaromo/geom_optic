\documentclass[a4paper]{article}
\usepackage[14pt]{extsizes} % для того чтобы задать нестандартный 14-ый размер шрифта
\usepackage{amsmath}
\usepackage[unicode, pdftex]{hyperref}
\usepackage[usenames]{color}
\usepackage[warn]{mathtext}
\usepackage[T2A]{fontenc}
\usepackage[utf8]{inputenc}
\usepackage[english, russian]{babel}
\usepackage{amsfonts}
\usepackage[left=20mm, top=15mm, right=15mm, bottom=15mm, nohead, footskip=10mm]{geometry} % настройки полей документа
\usepackage{graphicx}
\usepackage{wrapfig}
\usepackage{placeins}
\usepackage{float}
\usepackage{ucs}
\begin{document}
\author{Горяной Егор}
\date{7 июня 2022}
\title{Отчет по лабораторной работе "Геометрическая оптика"}
\maketitle


\section{Определение фокусного расстояния линзы}
Определим фокусное расстояние двумя способами:
\begin{enumerate}
    \item С помощью зеркала
    \item С помощью экрана
    \item С помощью метода Аббе
\end{enumerate}
\subsection{Способ 1}
Согласно формуле тонкой линзы:
\begin{equation}\label{eq1}
\frac{1}{F}=\frac{1}{d}+\frac{1}{f}
\end{equation}
В нашем случае $d=F$, тогда $f\to\infty$.\\
Таким образом, после выхода из линзы получим плоский волновой фронт. При отражении от зеркала и повторном прохождении
 через линзу, мы снова попадем в фокальную плоскость линзы, только изображение прдемета окажется перевернутым.\\
Соберем установку:\\
\begin{figure}[H]
    \centering
    \includegraphics[width=15cm, height=5cm]{IMG_20220603_132124.jpg}
    \caption{Установка с зеркалом}
\end{figure}
Тогда, получим, что $F=15$ см.
\subsection{Способ 2}
\begin{figure}[H]
    \centering
    \includegraphics[scale=0.5]{ris.png}
    \caption{Рисунок установки}
\end{figure}
Запишем систему уравнений:\\
\begin{equation*}
 \begin{cases}
   \frac{1}{F}=\frac{1}{d}+\frac{1}{f}~~~(1)
   \\
   \frac{1}{F}=\frac{1}{d'}+\frac{1}{f'}~~~(2)
   \\
    d+f=d'+f'=L~~~(3)
   \\
    d-d'=l~~~(4)
 \end{cases}
\end{equation*}
Приравнивая (1) и (2), а также используя, что $f+d=f'+d'$, получим: $fd=f'd'$\\
Подставим (4) в (3):\\
$f+d=f'+d-l\Longrightarrow f'-f=l$\\
Возведем (3) в квадрат, а также используем, что $fd=f'd'$:\\
$(d-d')(d+d')=(f'-f)(f'+f)$\\
Подставим ранее полученные равенства $d-d'=l,~f'-f=l$:\\
Получим ещё одно вспомогательное соотношение на расстояния:\\
$d+d'=f'+f$\\
Сложим его с (3):\\
$2d+d'+f=2f'+f+d'$\\
Откуда получим, что:\\
$f'=d$\\
Подставим это в $fd=f'd'$:\\
$d'=f$\\
\pagebreak\\
Тогда получили такие соотношения:\\
\begin{equation*}
 \begin{cases}
   f'=d~~~(5)\\
   d'=f~~~(6)
 \end{cases}
\end{equation*}
Подставим (6) в (4):\\
\begin{equation*}
 \begin{cases}
   f=d-l\\
   f+d=L
 \end{cases}
\end{equation*}
Из данной системы  получаем:\\
\begin{equation*}
 \begin{cases}
   f=\frac{L-l}{2}\\
   d=\frac{L+l}{2}
 \end{cases}
\end{equation*}
Подставим в (1) и выразим $F$:\\
$$F=\displaystyle\frac{L^2-l^2}{4L}$$
Соберем установку:\\
\begin{figure}[H]
    \centering
    \includegraphics[width=16cm, height=9cm]{IMG_20220603_132916.jpg}
    \caption{Установка с экраном}
\end{figure}
\\
\\
Запишем измерения:\vfill
\\
\begin{tabular}{|l|l|l|l|l|}
\hline
 $l$, см & $L$, см & $F$, см\\ \hline
 33.5&  74.8& 14.8\\ \hline
 46&  85.5& 15.2\\ \hline
 63&  100& 15.1\\ \hline
\end{tabular}
\quad Усредняя $F$: $F_{mean}=15$ см. \\
\subsection{Метод Аббе}
$$ \beta_1=\frac{y_1}{y}=\frac{F}{x_1},~\beta_2=\frac{y_2}{y}=\frac{F}{x_2} $$
Тогда разность расстояний от предмета до фокуса:\\
\\
$\Delta x=x_2-x_1=F(\frac{1}{\beta_1}-\frac{1}{\beta_2})\Longrightarrow$
$$\Longrightarrow F=\displaystyle\frac{\Delta x}{
    \displaystyle\frac{1}{\beta_2}-\displaystyle\frac{1}{\beta_1}}$$
\begin{figure}[H]
    \centering
    \includegraphics[scale=0.08]{IMG_20220603_142130.jpg}
    \caption{Установка по методу Аббе}
\end{figure}
\begin{figure}[H]
    \centering
    \includegraphics[scale=0.12]{IMG_20220603_142150.jpg}
    \caption{Увеличение изображения}
\end{figure}\\
\pagebreak\\
\hfill\\
$\Delta x=1.6$ см, $\beta_1=6.5,~\beta_2=3$\\
\\
Тогда $F\approx14.1$ см.\\

\end{document}
